\documentclass[a4paper]{article}

%% Language and font encodings
\usepackage[english]{babel}
\usepackage[utf8x]{inputenc}
\usepackage[T1]{fontenc}

%% Sets page size and margins
\usepackage[a4paper,top=3cm,bottom=2cm,left=3cm,right=3cm,marginparwidth=1.75cm]{geometry}

%% Useful packages
\usepackage{amsmath}
\usepackage{graphicx}
\usepackage{indentfirst}
\usepackage[colorinlistoftodos]{todonotes}
\usepackage[colorlinks=true, allcolors=blue]{hyperref}

\title{Projeto Yggdrasil - Parte 1}
\author{Beatriz Marouelli, Leonardo Lana}

\begin{document}
\maketitle

\section*{I - Complementando o Modelo Geral}

A primeira parte do EP consiste em completar o modelo do sistema Yggdrasil feito em aula, acrescentando o controle de acesso.

\section*{II - Descrevendo os Componentes do Modelo Geral}

\subsection*{Entidade Regular Usuário}

% A entidade regular Usuário armazena os usuários que estão cadastrados no sistema.
% Ela possui o atributo Login, que é chave primária e portanto deverá ser NOTNULL,
% e o atributo Senha, que por liberar o acesso a um perfil e um conjunto de serviços
% associados a esse perfil, não poderá ter valor NULL.

A entidade regular Usuário armazena os usuários que estão cadastrados no sistema.
Ela possui o atributo Login, que é chave primária e portanto deverá ser NOTNULL,
e o atributo Senha, que por liberar o acesso ao sistema não poderá ter valor NULL.

\begin{itemize}
    \item Login (chave primária)
    \item Senha
\end{itemize}

\subsection*{Entidade Regular Perfil}

% A entidade regular Perfil armazena os tipos de perfil através dos quais os
% usuários terão acesso aos serviços do sistema, sendo possível um usuário possuir
% mais de um perfil. Essa entidade é composta por Nome, que será um código
% de identificação do perfil e por ser chave primária deverá ser NOTNULL. (Além de
% Tipo\_Perfil a entidade terá um atributo Descrição, que irá conter uma pequena
% explicação dos serviços disponíveis.

A entidade regular Perfil armazena os tipos de perfil através dos quais os
usuários terão acesso aos serviços do sistema, sendo possível um usuário possuir
mais de um perfil. Essa entidade é composta por Nome, que será um código
de identificação do perfil e por ser chave primária deverá ser NOTNULL.

\begin{itemize}
    \item Nome (chave primária)
\end{itemize}

\subsection*{Entidade Regular Serviço}

A entidade regular Serviço armazena os tipos de serviço que estão habilitados no
sistema. Através desses serviços será possível acessar ou manipular os dados
armazenados, desde que o perfil permita a utilização do serviço. Ela é formada
por: Nome, que terá um código de identificação diferente pra cada serviço
e não poderá assumir o valor NULL por ser chave primária; Descrição, que irá conter
a explicação de quais são os efeitos do serviço no sistema.

\begin{itemize}
    \item Nome (chave primária)
    \item Descrição
\end{itemize}

\subsection*{Entidade Regular Aluno}

% A entidade regular Aluno armazena os dados de cada instância de um aluno do
% Bacharelado em Ciência da Computação, que tenha sido cadastrado (ou inserido no
% banco?). Os atributos que compõem a entidade são os seguintes: Nome, que não
% poderá ser do tipo NULL; N° USP, que é chave primária; Data\_Nascimento e Sexo.

A entidade regular Aluno armazena os dados de cada instância de um aluno do
Bacharelado em Ciência da Computação, que tenha sido cadastrado. Os atributos que
compõem a entidade são os seguintes: NUSP, que é chave primária, portanto NOTNULL;
Nome, que não poderá ser do tipo NULL; Data de Ingresso, que não pode ser NULL,
e Data de Formatura, que pode ser NULL, para modelar os casos em que um aluno
saiu do curso antes de se formar.

\begin{itemize}
    \item NUSP (chave primária)
    \item Nome
    \item Data de Ingresso
    \item Data de Formatura
\end{itemize}

\subsection*{Entidade Regular Professor}

% A entidade regular Professor armazena os dados de cada instância de um Professor
% do curso Bach. em Ciência da Computação que tenha se cadastrado no sistema. O
% professor poderá acompanhar alunos, se estes concordarem, e sugerir planos que
% incluam disciplinas preferenciais para o desenvolvimento em uma área de pesquisa
% ou o cumprimento de uma trilha. Além disso, ele pode acessar boa parte dos dados
% no banco, para verificar a procura de uma disciplina pelos alunos, por exemplo.
% Ela é composta por Nome, que não poderá ser do tipo NULL; N° USP, que é chave
% primária; Área\_Pesquisa, que identificará a área de estudo do professor;
% Data\_Nascimento e Sexo.

A entidade regular Professor armazena os dados de cada instância de um professor
do Departamento de Ciência da Computação que tenha se cadastrado no sistema. O
professor poderá acompanhar alunos, se estes concordarem, e sugerir planos que
incluam disciplinas preferenciais para o desenvolvimento em uma área de pesquisa
ou o cumprimento de uma trilha. Além disso, ao professor estará associado o
perfil que permite ele acessar boa parte dos dados no banco, para verificar a
procura de uma disciplina pelos alunos, por exemplo. Ela é composta por NUSP, que é
chave primária;  Nome, que não poderá ser do tipo NULL; Área de Pesquisa, que
identificará a área de estudo do professor.

\begin{itemize}
    \item NUSP (chave primária)
    \item Nome
    \item Área de Pesquisa
\end{itemize}

\subsection*{Entidade Regular Disciplina}

% A entidade regular Disciplina armazena os dados de cada instância de disciplina
% oferecida por algum(a) Instituto(Escola) da USP e que estejam disponíveis para
% o BCC.\ Ela é composta pelos seguintes atributos: Código, que é chave primária;
% Descrição, contendo uma explicação do que é a disciplina; Local, que irá assumir
% a sigla do Instituto ou Escola que oferece a disciplina; Ementa, que irá conter
% uma descrição mais técnica dos assuntos abordados, bibliografia e critérios de
% avaliação; Créditos, que indicará o número de créditos atribuídos à disciplina;
% Semestre, que indicará em qual semestre a disciplina será oferecida.

A entidade regular Disciplina armazena os dados de cada instância de disciplina
oferecida por algum(a) Instituto(Escola) da USP e que estejam disponíveis para
o BCC. Ela é composta pelos seguintes atributos: Código, que é chave primária;
Local, que irá assumir a sigla do Instituto ou Escola que oferece a disciplina;
Ementa, que irá conter uma descrição mais técnica dos assuntos abordados,
bibliografia e critérios de avaliação; Créditos-Aula, que indicará o número de créditos atribuídos à disciplina; Semestre, que indicará em qual semestre a disciplina será oferecida.

Em particular, cada disciplina irá se especializar em obrigatória, optativa
optativa eletiva e optativa livre. Por ser uma especialização
total e disjunta, cada disciplina terá um atributo Aproveitamento, que modelerá
o código que representa para qual dos três tipos a disciplina irá se
especializar.

\begin{itemize}
    \item Código (chave primária)
    \item Nome
    \item Ementa
    \item Créditos-aula
    \item Créditos-trabalho
    \item Semestre
    \item Aproveitamento
\end{itemize}

\subsection*{Entidade Regular Módulo}

A entidade regular Módulo armazena os dados de cada instância de um módulo de
uma trilha, que é um conjunto de disciplinas que fazem parte de uma trilha. Cada
módulo define o conjunto de disciplinas que um aluno deve cumprir para progredir
na trilha que quer completar. O Módulo terá Nome, que será a chave primária,
portato será NOTNULL.

\begin{itemize}
    \item Nome (chave primária)
\end{itemize}

\subsection*{Entidade Regular Trilha}
A entidade regular Trilha armazena os dados de cada instância de uma trilha, que
será um conjunto de módulos de disciplinas. Cada trilha defina uma
especialização para o aluno do BCC. A Trilha terá Nome, que será a chave
primária, portanto será NOTNULL.

\begin{itemize}
    \item Nome (chave primária)
\end{itemize}


\subsection*{Relacionamento Possui}
O relacionamento Possui  modela a conexão entre as entidades regulares Usuário e
Perfil. Essa relação permite que um Usuário tenha mais de uma Perfil, e que um
Perfil pode ser possuido por mais de um Usuário.  A conexão entre essas duas
entidades define qual o nível de autorização o Usuário terá dentro do sistema.

\subsection*{Relacionamento Usa}
O relacionamento Usa modela a conexão entre as entidade regulares Perfil e
Serviço. Essa relação permite que um Serviço esteja associado a mais de um
Perfil e que um Perfil use múltiplos Serviços. A conexão entre essas dudas
entidades define quais operações determinado Perfil pode executar no sistema.

\subsection*{Relacionamento Tutora}
O relacionamento Tutora modela a conexão entre as entidades regulares Aluno e
Professor. Essa relação permite que um Aluno tenha mais que um Professor como
seu tutor, e que um Professor tutore vários alunos. A conexão entre essas duas
entidades permite que um Professor visualize e sugira planejamento para Alunos,
os quais ele tutora.

\subsection*{Relacionamento Planeja Cursar}
O relacionamento Planeja Cursar modela a conexão entre as entidades regulares Aluno e
Disciplina. Essa relação permite que um Aluno planeje cursar diversas matérias,
e que uma matéria tenha vários alunos planejando cursar ela.

Este relacionamento terá alguns atributos como: Semestre e Ano, que modelam em
qual ano e semestre o Aluno pretende fazer a matéria; Aproveitamento, que dirá
como o Aluno pretende aproveitar os créditos daquela matéria; e Aprovado? que
modela se uma aluno foi ou não aprovado na matéria.

Se o valor de Aprovado? for verdadeiro, então os créditos da matéria serão
contabilizados na categoria escolhida em Aproveitamento.

Nenhum dos atributos podem ser NULL, e Aprovado? terá o valor padrão de FALSE.

\begin{itemize}
    \item Semestre
    \item Ano
    \item Aproveitamento
    \item Aprovado?
\end{itemize}

\subsection*{Relacionamento Forma}
O relacionamento Forma modela a conexão entre as entidades regulares Módulo e
Disciplina. Essa relação permite que um Módulo é formado por várias Disciplina,
e que uma Disciplina faça parte de vários módulos.

\subsection*{Relacionamento Pertence}
O relacionamento Pertence modela a conexão entre as entidades regulares Módulo e
Trilha. Essa relação permite que um Módulo pertença a várias trilhas, e que cada
Trilha tenha vários Módulos.

Este relacionamento terá um atributo: Minímo, que modela a quantidade miníma de
matérias que um aluno deve cumprir para completar aquele módulo em determinada
trilha. Este atributo foi colocado na relação, pois o minímo de matérias
necessárias muda conforme a trilha na qual o módulo está inserido. Este atributo
não poderá ser NULL.

\begin{itemize}
    \item Minímo
\end{itemize}

\section*{III - Restrições de Domínio}

\section*{IV - Funcionalidades Esperadas}

\section*{V - Diagrama Estendido}

\section*{VI - Modelo Lógico}



\end{document}
